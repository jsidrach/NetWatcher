\chapter{Pruebas\label{cap:pruebas}}

TODO: [Introducción]
- Por qué han sido importantes
- Quién las ha realizado
  - Estudiante
    - Otros miembros del grupo de investigación (potenciales usuarios)
- División entre pruebas de verificación y pruebas de validación

\section{Pruebas de verificación\label{sec:pb:verificacion}}

TODO: Pruebas de verificación
- Qué son
- Plan de pruebas: todas las pruebas que se han realizado
- Tipos de pruebas

\subsection*{Inspección del código\label{ssec:pb:inspeccion}}

La inspección del código es una técnica que consiste en revisar el código fuente.
Se ha aplicado en todos los archivos de código del proyecto, con el objetivo mejorar la estructura interna y el estilo del código fuente.
Sin embargo, no se ha utilizado para descubrir errores, ya que para eso se ha recurrido a otro tipo de pruebas, como las de caja negra y las de caja blanca.

La estrategia seguida ha sido realizar una inspección en cada fichero, una vez ha pasado al menos una semana desde su codificación.
Así, al no tenerla tan cercana, se ha facilitado el evaluar de manera más objetiva la implementación.
Como consecuencia de estas inspecciones, se ha refactorizado parte del código, mejorando su estructura interna, claridad y mantenibilidad.

\subsection*{Pruebas de caja blanca\label{ssec:pb:caja_blanca}}

Las pruebas de caja blanca son las que se realizan teniendo en cuenta la estructura interna del programa.
Tienen como objetivo verificar el correcto funcionamiento del código y detectar errores.
Concretamente, se han llevado a cabo aquellas basadas en la comprobación de los posibles flujos de ejecución de una función, mediante su invocación con valores típicos, aleatorios y límites.

Debido al alto coste temporal de plantear y ejecutar este tipo de pruebas para todo el sistema, se ha decidido aplicarlas en un único submódulo, considerado crítico para la aplicación: la máquina de estados finitos que formaliza el estado de la \gls{FPGA}.
Como resultado de estas pruebas, se han corregido errores menores que se producían en su mayoría con valores extremos.

\subsection*{Pruebas de caja negra\label{ssec:pb:caja_negra}}

Las pruebas de caja negra, en contraposición con las de caja blanca, son aquellas que se realizan sin tener conocimiento de la estructura interna del programa.
Se ha utilizado este tipo de pruebas
Estas pruebas la ejecución
pruebas de caja blanca
    Estrategia ascendente, separando back-end y front-end
Back-End
unitarias
common > utils > funciones de la API (POSTMAN)
    - Fácilmente replicables, test de pruebas exportado/importable a la app
        - Postman, cada uno de los métodos de la API pública del back-end, que hacían uso de los métodos básicos
Front-End
    - Proxy
    - Modelo MVC

    Resultado


\subsection*{Pruebas de integración\label{ssec:pb:integracion}}
  - Prueba de integración
    Back-end y front-end
    Resultado

\subsection*{Pruebas sobre la interfaz de usuario\label{ssec:pb:interfaz}}

    - para cualquier estado de la sonda
    - cambios esperados en la interfaz
    - valores límite en los formularios disponibles

\section{Pruebas de validación\label{sec:pb:validacion}}

TODO: Pruebas de validación
- Qué son

- Comprobación de los requisitos
  - Se cumplen todos los funcionales, cómo se comprueba
  - Se cumplen todos los no funcionales, cómo se comprueban
