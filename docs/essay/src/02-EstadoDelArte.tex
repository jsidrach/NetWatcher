\chapter{Estado del arte\label{cap:estadoDelArte}}

TODO: [Introducción]


Sistemas de captura y/o reproducción de tráfico de red y sistemas similares o cercanos al que se propone

\section{tcpdump y libpcap\label{sec:eda:tcpdump}}

La utilidades \textit{tcpdump} y \textit{libpcap}~\cite{tcpdump}, implementadas como librerías, permiten analizar el tráfico que circula por la red.
Así, el usuario puede capturar y mostrar en tiempo real los paquetes transmitidos y recibidos en una red a la que el ordenador esté conectado.
Adicionalmente, da la posibilidad de aplicar filtros sobre la salida.
Se suele utilizar para depurar aplicaciones que envían y reciben tráfico de red, aunque tiene otros usos como leer datos no cifrados enviados por otros ordenadores.

Pros
- Open Source
- Herramienta establecida

Cons
- Línea de comandos
- Se maneja sobre el mismo PC
- No da la tasa

\section{Wireshark\label{sec:eda:wireshark}}

\textit{Wireshark}~\cite{wireshark} es un analizador de protocolos de red.
Provee una funcionalidad similar a la de \textit{tcpdump}, añadiendo una interfaz gráfica y más opciones de filtrado de la información.
Así, permite o bien ver todo el tráfico en tiempo real que pasa a través de una red (almacenándolo opcionalmente), o bien analizar un tráfico que fue capturado anteriormente.


PROS:
Wireshark has a rich feature set which includes the following:
Deep inspection of hundreds of protocols, with more being added all the time
Live capture and offline analysis
Standard three-pane packet browser
Multi-platform: Runs on Windows, Linux, OS X, Solaris, FreeBSD, NetBSD, and many others
Captured network data can be browsed via a GUI, or via the TTY-mode TShark utility
The most powerful display filters in the industry
Rich VoIP analysis
Read/write many different capture file formats: tcpdump (libpcap), Pcap NG, Catapult DCT2000, Cisco Secure IDS iplog, Microsoft Network Monitor, Network General Sniffer® (compressed and uncompressed), Sniffer® Pro, and NetXray®, Network Instruments Observer, NetScreen snoop, Novell LANalyzer, RADCOM WAN/LAN Analyzer, Shomiti/Finisar Surveyor, Tektronix K12xx, Visual Networks Visual UpTime, WildPackets EtherPeek/TokenPeek/AiroPeek, and many others
Capture files compressed with gzip can be decompressed on the fly
Live data can be read from Ethernet, IEEE 802.11, PPP/HDLC, ATM, Bluetooth, USB, Token Ring, Frame Relay, FDDI, and others (depending on your platform)
Decryption support for many protocols, including IPsec, ISAKMP, Kerberos, SNMPv3, SSL/TLS, WEP, and WPA/WPA2
Coloring rules can be applied to the packet list for quick, intuitive analysis
Output can be exported to XML, PostScript®, CSV, or plain text

Cons
- Interfaz no muy intuitiva
- No da la tasa
- Se maneja sobre el mismo PC

\section{Detect-Pro\label{sec:eda:detectpro}}

\textit{Detect-Pro}~\cite{detectpro} es una herramienta modular de análisis pasivo (sin intervenir).
Analiza todo el tráfico de un enlace de red paquete a paquete, proporcionando series temporales de tráfico, análisis a nivel de flujo, análisis de tendencias y recolección selectiva de \glspl{traza}.
Identifica así patrones de actividad, detectando anomalías mediante estos mismos patrones.

Pros
- Mucha funcionalidad
- Procesar flujos (a tasa de linea, 10GB)

Cons
- Privativo
- Complejo
- 10GB Solo si el numero de flujos no es muy elevado

\section{The Open Source Network Tester\label{sec:eda:osnt}}

\textit{The Open Source Network Tester (OSNT)}~\cite{osnt} es un generador y capturador de tráfico.
Se ejecuta sobre cuatro \textit{NetFPGA-10G} (el mismo modelo de sonda que el seleccionado en este proyecto).
Permite generar o capturar paquetes de todo tipo de tamaño, y establecer la tasa a la que se realizan estas operaciones.

Pros

Cons
- No da la tasa
- Interfaz mala (no intuitiva, linux)
- Se maneja sobre el mismo PC

\section{Conclusiones\label{sec:eda:conclusiones}}

TODO: Conclusiones
