\chapter{Introducción}

TODO: Introducción del trabajo


\section{\'Ambito}

Trabajo de Fin de Grado

 sondas de red de altas prestaciones

grupo de investigación \textit{High Performance Computing and Networking}. 

TODO: Ámbito del trabajo


\section{Motivación}

Necesaria simplificación gestión de sondas de red de altas prestaciones por medio de una interfaz gráfica

ampliar la funcionalidad ofreciendo monitorización 
TODO: Motivación del trabajo


\section{Objetivos}

el objetivo de este Trabajo de Fin de Grado es diseñar, implementar y validar una interfaz web para la gestión de sondas de red de altas prestaciones 

gestionar \glspl{traza} 

monitorizar

estadísticas del sistema

registrar eventos


TODO: Objetivos del trabajo


\section{Estructura del documento}

En el capítulo \ref{cap:estadoDelArte} se realiza un análisis del estado del arte. Se analizan tanto los sistemas de captura y reproducción de tráfico web existentes como las interfaces de gestión y monitorización de estos sistemas, para posteriormente extraer conclusiones sobre lo estudiado.

En el capítulo \ref{cap:defProyecto} se define la aplicación que se va a diseñar, así como la metodología seguida y las herramientas utilizadas en el proyecto.
En el capítulo \ref{cap:requisitos} se describen los requisitos funcionales y no funcionales de la aplicación.

En el capítulo \ref{cap:disenho} se formaliza el diseño de la aplicación a implementar, comentando la arquitectura de la aplicación y los módulos en los que se divide.
En el capítulo \ref{cap:implementacion} se documenta la implementación de la aplicación, estructurada en dos partes bien diferenciadas: \gls{back-end} y \gls{front-end}.
En el capítulo \ref{cap:pruebas} se explica el proceso de pruebas seguido para la verificación y validación de la aplicación construida, comprobando así el correcto funcionamiento de la misma.
En el capítulo \ref{cap:mantenimiento} se espeficica cómo se va a realizar el mantenimiento de la aplicación.

En el capítulo \ref{cap:conclusiones} se exponen las conclusiones finales sobre el trabajo realizado. Por último, en el capítulo \ref{cap:lineasDeTrabajoFuturo} se plantean posibles líneas de trabajo futuro que podrían ser abordadas con el objetivo de mejorar y ampliar diferentes aspectos de la aplicación desarrollada.