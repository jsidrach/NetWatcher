\chapter{Verificación y validación\label{cap:pruebas}}

En este capítulo se explica el proceso seguido para la verificación y validación de la aplicación.
Este proceso tiene como objetivo asegurar la corrección del sistema, comprobar que satisface los requisitos y mejorar la implementación evaluada.

\section{Verificación\label{sec:pb:verificacion}}

La verificación del proyecto busca comprobar que la aplicación concuerda con su especificación, es decir, que la implementación es correcta.
Dentro de este ámbito, se han planteado y realizado inspecciones del código, pruebas de caja blanca, pruebas de caja negra, pruebas de integración y pruebas sobre la interfaz de usuario.

\subsection*{Inspección del código\label{ssec:pb:inspeccion}}

La inspección del código es una técnica que consiste en revisar el código fuente.
Se ha aplicado en todos los archivos de código del proyecto, con el objetivo mejorar la estructura interna y el estilo del código fuente.
Sin embargo, no se ha utilizado para descubrir errores, ya que para eso se ha recurrido a otro tipo de pruebas, como las de caja negra y las de caja blanca.

La estrategia seguida ha sido realizar una inspección en cada fichero, una vez ha pasado al menos una semana desde su codificación.
Así, al no tenerla tan cercana, se ha facilitado el evaluar de manera más objetiva la implementación.

\subsection*{Pruebas de caja blanca\label{ssec:pb:caja_blanca}}

Las pruebas de caja blanca son las que se realizan teniendo en cuenta la estructura interna del programa.
Tienen como objetivo verificar el correcto funcionamiento del código y detectar errores.
Concretamente, se han llevado a cabo aquellas basadas en la comprobación de los posibles flujos de ejecución de una función, mediante su invocación con valores típicos, aleatorios y límite.

Debido al alto coste temporal de plantear y ejecutar este tipo de pruebas para todo el sistema, se ha decidido aplicarlas en un único submódulo, considerado crítico para la aplicación: la máquina de estados finitos que formaliza el estado de la \gls{FPGA}.
De esta forma se ha comprobado, mediante pruebas más exhaustivas que las de caja negra, que la parte esencial de la aplicación tenga el comportamiento esperado.

\subsection*{Pruebas de caja negra\label{ssec:pb:caja_negra}}

Las pruebas de caja negra, en contraposición con las de caja blanca, son aquellas que se realizan sin tener en cuenta la estructura interna del programa.
Para este tipo de pruebas, se ha seguido una estrategia ascendente, comprobando primero las funciones de más bajo nivel.
Dada la arquitectura de la aplicación, se ha decidido separar las pruebas de caja negra entre el \gls{back-end} y el \gls{front-end}.

Dentro de los componentes \gls{back-end}, se han realizado primero pruebas en los métodos básicos, comunes a todos los módulos (\textit{common.js}).
Posteriormente, se han comprobado las funciones auxiliares de cada módulo (\textit{\_utils.js}).
Por último, se ha creado y ejecutado un conjunto de pruebas sobre la \gls{API} del \gls{servicioweb} \gls{FPGA} con \textit{Postman}~\cite{postman}.
Esto permite que se puedan replicar fácilmente, ya que pueden ser exportadas a un único fichero de configuración para después importarlo cuando se deseen volver a ejecutar las pruebas.

Para el \gls{front-end}, se ha elegido efectuar pruebas de caja negra sobre un único módulo, ya que se han considerado de más utilidad aplicar pruebas de contenido sobre la interfaz de usuario (ver apartado \ref{ssec:pb:interfaz}).
De este modo, se ha verificado el funcionamiento del \gls{proxy} haciendo uso de las mismas pruebas de \textit{Postman} creadas para el \gls{front-end}, comprobando que se obtenía un resultado idéntico.

\subsection*{Pruebas de integración\label{ssec:pb:integracion}}

Las pruebas de integración tienen como objetivo verificar que los componentes de la aplicación encajan correctamente entre sí.
Tras las pruebas de caja blanca y caja negra, se ha realizado este tipo de pruebas de forma ascendente, empezando con los componentes más básicos y terminando con la integración entre el \gls{back-end} y el \gls{front-end}.

\subsection*{Pruebas sobre la interfaz de usuario\label{ssec:pb:interfaz}}

Para la interfaz web se han realizado pruebas basadas en el contenido.
En primer lugar, se ha verificado que la interfaz web se encuentra disponible e informa correctamente al usuario del estado del sistema, incluso cuando el \gls{back-end} no está operativo.
En segundo lugar, se ha comprobado que se producen los cambios esperados en la interfaz como respuesta a las acciones del usuario.
Finalmente, se han evaluado todos los formularios en la interfaz rellenando los parámetros disponibles con valores válidos e inválidos, constatando que el usuario recibe información sobre qué parámetros son correctos y cuáles no.

\section{Validación\label{sec:pb:validacion}}

La validación de la aplicación consiste en constatar que satisface el propósito para el que fue planteada.
Se comprueba, por tanto, que cumpla los requisitos especificados en el capítulo~\ref{cap:requisitos}, tanto funcionales como no funcionales.

En este contexto, se ha confirmado que existe una forma de satisfacer, mediante la interfaz web, cada uno de los requisitos funcionales.
Respecto a los requisitos no funcionales, se ha validado uno a uno su cumplimiento, ya que estaban todos relacionados con la interfaz web.
Adicionalmente, se ha evaluado la usabilidad de la aplicación comprobando que un grupo de usuarios pudiese utilizar toda la funcionalidad ofrecida por interfaz, con la ayuda opcional del manual.

Con este propósito se ha determinado, mediante un formulario de respuesta múltiple, con qué facilidad los usuarios pueden completar tareas desde la interfaz (ver Tabla~\ref{sec:pb:test}).
Han colaborado, para ello, personas pertenecientes al grupo de investigación en el que se enmarca este proyecto.
Dado que para realizar pruebas se cuenta con una única sonda de red, cada test solo puede comenzar una vez ha terminado el anterior.
Por este motivo, se ha decidido reducir el conjunto de tareas a las más significativas, disminuyendo el tiempo necesario.
Destacar que no se especifica en ningún sitio la página concreta a la que debe acceder el usuario para completar cada acción requerida.

La dificultad con la que se completa cada tarea se clasifica en una de las siguientes categorías:
\begin{itemize}
  \item \textbf{Fácil}: el usuario ha completado la tarea sin consultar el manual, en menos de un minuto.
  \item \textbf{Media}: el usuario ha completado la tarea sin consultar el manual, tardando más de un minuto para ello.
  \item \textbf{Difícil}: el usuario ha completado la tarea consultando el manual.
  \item \textbf{No lograda}: el usuario ha desistido y no ha conseguido completar la tarea.
\end{itemize}

\begin{table}[!htp]
\resizebox{\textwidth}{!}{%
\begin{tabular}{l|l|l|l|l|}
\cline{2-5}
\multicolumn{1}{c|}{} & \multicolumn{4}{c|}{\cellcolor[HTML]{EFEFEF}{\bf Dificultad}} \\ \hline
\multicolumn{1}{|c|}{\cellcolor[HTML]{EFEFEF}{\bf Tarea}} & \cellcolor[HTML]{CAFFC9}{\color[HTML]{333333} {\bf Fácil}} & \cellcolor[HTML]{FFFFC7}{\color[HTML]{333333} {\bf Media}} & \cellcolor[HTML]{FFCCC9}{\color[HTML]{333333} {\bf Difícil}} & \cellcolor[HTML]{333333}{\color[HTML]{FFFFFF} {\bf No lograda}} \\ \hline
\multicolumn{1}{|l|}{Configurar la interfaz para conectarse a una \gls{URL}} &  &  &  &  \\ \hline
\multicolumn{1}{|l|}{Programar la sonda en modo reproducción} &  &  &  &  \\ \hline
\multicolumn{1}{|l|}{Reproducir una \gls{traza} en bucle, por el puerto 0 y IFG 0} &  &  &  &  \\ \hline
\multicolumn{1}{|l|}{Parar la reproducción de la \gls{traza}} &  &  &  &  \\ \hline
\multicolumn{1}{|l|}{Reprogramar la sonda en modo captura} &  &  &  &  \\ \hline
\multicolumn{1}{|l|}{Ordenar capturar una traza de 200 GB por el puerto 3} &  &  &  &  \\ \hline
\multicolumn{1}{|l|}{Parar la captura de la \gls{traza}} &  &  &  &  \\ \hline
\multicolumn{1}{|l|}{Convertir una \gls{traza} \gls{simple} a formato \gls{pcap}} &  &  &  &  \\ \hline
\multicolumn{1}{|l|}{Renombrar una \gls{traza}} &  &  &  &  \\ \hline
\multicolumn{1}{|l|}{Borrar una \gls{traza}} &  &  &  &  \\ \hline
\multicolumn{1}{|l|}{Comprobar el espacio de almacenamiento disponible} &  &  &  &  \\ \hline
\multicolumn{1}{|l|}{Comprobar la velocidad de escritura del \gls{RAID}} &  &  &  &  \\ \hline
\end{tabular}
}
\caption{Conjunto de preguntas que conforman el test de validación.}
\label{sec:pb:test}
\end{table}

\section{Resultados\label{sec:pb:resultados}}

Las distintas pruebas de verificación han tenido las siguientes consecuencias directas:
\begin{itemize}
  \item Se ha refactorizado parte de la implementación, mejorando su estructura interna, claridad y mantenibilidad, fruto de las inspecciones de código.
  \item Se han corregido errores menores gracias a las pruebas de caja blanca y de caja negra.
  \item Se han podido detectar y subsanar errores menores relacionados con la comunicación entre los componentes, como resultado de las pruebas de integración.
Otra consecuencia ha sido el descubrimiento y la corrección del hecho de que funcionalidad no esencial implementada en el \gls{back-end} no tuviese correspondencia en el \gls{front-end}, por lo que nunca se llegaba a aprovechar.
\end{itemize}

Respecto a las pruebas de validación, el formulario ha permitido comprobar cómo de fácil es manejar la interfaz web.
Los resultados han sido satisfactorios, ya que si bien no todas las tareas han sido clasificadas como ``Fácil'', ninguna de ellas no ha podido ser completada por el usuario.
Por lo general, la gran mayoría de tareas ha tenido una dificultad ``Fácil'' o ``Media'', por lo que no han requerido la ayuda del manual de usuario.
Se valida así el cumplimiento de uno de los objetivos del proyecto, simplificar la gestión de la sonda de red.
